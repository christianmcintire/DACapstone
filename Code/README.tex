% Options for packages loaded elsewhere
\PassOptionsToPackage{unicode}{hyperref}
\PassOptionsToPackage{hyphens}{url}
%
\documentclass[
]{article}
\usepackage{amsmath,amssymb}
\usepackage{iftex}
\ifPDFTeX
  \usepackage[T1]{fontenc}
  \usepackage[utf8]{inputenc}
  \usepackage{textcomp} % provide euro and other symbols
\else % if luatex or xetex
  \usepackage{unicode-math} % this also loads fontspec
  \defaultfontfeatures{Scale=MatchLowercase}
  \defaultfontfeatures[\rmfamily]{Ligatures=TeX,Scale=1}
\fi
\usepackage{lmodern}
\ifPDFTeX\else
  % xetex/luatex font selection
\fi
% Use upquote if available, for straight quotes in verbatim environments
\IfFileExists{upquote.sty}{\usepackage{upquote}}{}
\IfFileExists{microtype.sty}{% use microtype if available
  \usepackage[]{microtype}
  \UseMicrotypeSet[protrusion]{basicmath} % disable protrusion for tt fonts
}{}
\makeatletter
\@ifundefined{KOMAClassName}{% if non-KOMA class
  \IfFileExists{parskip.sty}{%
    \usepackage{parskip}
  }{% else
    \setlength{\parindent}{0pt}
    \setlength{\parskip}{6pt plus 2pt minus 1pt}}
}{% if KOMA class
  \KOMAoptions{parskip=half}}
\makeatother
\usepackage{xcolor}
\usepackage[margin=1in]{geometry}
\usepackage{graphicx}
\makeatletter
\def\maxwidth{\ifdim\Gin@nat@width>\linewidth\linewidth\else\Gin@nat@width\fi}
\def\maxheight{\ifdim\Gin@nat@height>\textheight\textheight\else\Gin@nat@height\fi}
\makeatother
% Scale images if necessary, so that they will not overflow the page
% margins by default, and it is still possible to overwrite the defaults
% using explicit options in \includegraphics[width, height, ...]{}
\setkeys{Gin}{width=\maxwidth,height=\maxheight,keepaspectratio}
% Set default figure placement to htbp
\makeatletter
\def\fps@figure{htbp}
\makeatother
\setlength{\emergencystretch}{3em} % prevent overfull lines
\providecommand{\tightlist}{%
  \setlength{\itemsep}{0pt}\setlength{\parskip}{0pt}}
\setcounter{secnumdepth}{-\maxdimen} % remove section numbering
\ifLuaTeX
  \usepackage{selnolig}  % disable illegal ligatures
\fi
\usepackage{bookmark}
\IfFileExists{xurl.sty}{\usepackage{xurl}}{} % add URL line breaks if available
\urlstyle{same}
\hypersetup{
  pdftitle={README},
  pdfauthor={Christian McIntire},
  hidelinks,
  pdfcreator={LaTeX via pandoc}}

\title{README}
\author{Christian McIntire}
\date{2025-03-15}

\begin{document}
\maketitle

\section{\texorpdfstring{\textbf{Introduction}}{Introduction}}\label{introduction}

Housing affordability remains a pressing issue across the United States,
particularly in regions experiencing rapid urbanization and population
growth. The Southeastern United States has seen significant population
growth in recent years, whereas the Midwestern United States has
remained much more stable or even seen population decline
(Biernacka-Lievestro \& Fall, 2023).

This study seeks to examine the relationship between population density,
growth, and housing prices in these two regions, to determine how
demographic shifts impact affordability. The Southeastern and Midwestern
regions are defined as follows:

\begin{itemize}
\tightlist
\item
  \textbf{Southeastern U.S.}: Arkansas, Louisiana, Kentucky, Tennessee,
  Mississippi, Alabama, Georgia, Florida, North Carolina, South
  Carolina, Virginia, and West Virginia.
\item
  \textbf{Midwestern U.S.}: Minnesota, Wisconsin, Michigan, Ohio,
  Indiana, Illinois, Iowa, and Missouri.
\end{itemize}

\subsubsection{\texorpdfstring{\textbf{Research
Questions}}{Research Questions}}\label{research-questions}

\begin{enumerate}
\def\labelenumi{\arabic{enumi}.}
\tightlist
\item
  How has population density historically affected housing prices in the
  Southeastern and Midwestern United States?
\item
  How can these patterns be measured to predict future changes in the
  U.S. housing market?
\end{enumerate}

\begin{center}\rule{0.5\linewidth}{0.5pt}\end{center}

\section{\texorpdfstring{\textbf{Methodology}}{Methodology}}\label{methodology}

This study employs the following techniques:

\begin{itemize}
\tightlist
\item
  \textbf{Choropleth Maps} -- Visualizing county-level home prices using
  spatial analysis.
\item
  \textbf{Hedonic Regression Models} -- Decomposing housing prices into
  key factors, including regional fixed effects and population density.
\item
  \textbf{Panel Data Estimation} -- Handling time-series data with
  multiple geographic units.
\item
  \textbf{ARIMA Forecasting} -- Time-series forecasting to predict
  future housing prices.
\end{itemize}

\subsubsection{\texorpdfstring{\textbf{Data
Sources}}{Data Sources}}\label{data-sources}

This project uses publicly available datasets from:

\begin{enumerate}
\def\labelenumi{\arabic{enumi}.}
\tightlist
\item
  \textbf{National Association of Realtors} -- County-level median home
  prices (2024).
\item
  \textbf{Federal Reserve Bank of St.~Louis (FRED)} -- Housing Price
  Index data (2024).
\item
  \textbf{U.S. Census Bureau} -- County-level population estimates
  (2024).
\item
  \textbf{Zillow Research Data} -- Historical home values and unique
  price metrics (2024).
\item
  \textbf{TIGER/Line Shapefiles} from the \textbf{U.S. Census Bureau} --
  Geographic county/state boundaries.

  \begin{itemize}
  \tightlist
  \item
    \textbf{Shapefiles were too large for GitHub but can be downloaded
    here:}\\
    🔗 \href{https://www2.census.gov/geo/tiger/TIGER2024/}{U.S. Census
    Bureau Shapefiles}
  \end{itemize}
\end{enumerate}

\begin{center}\rule{0.5\linewidth}{0.5pt}\end{center}

\subsubsection{Key Findings}\label{key-findings}

\begin{itemize}
\tightlist
\item
  Population trends in the \textbf{Southeastern U.S.} have shown
  significant growth, whereas the \textbf{Midwestern U.S.} has remained
  more stable or declined in some areas.
\item
  Housing prices in the \textbf{Southeast} have seen steeper increases,
  especially in urban and suburban areas.
\item
  \textbf{Choropleth maps} reveal strong regional patterns in home
  prices, with coastal and metropolitan counties exhibiting the highest
  median prices.
\item
  \textbf{ARIMA Forecasting} suggests continued home price growth in
  both regions, albeit at different rates.
\end{itemize}

\subsubsection{Next Steps}\label{next-steps}

\paragraph{1. Hedonic Regression Model}\label{hedonic-regression-model}

\begin{itemize}
\tightlist
\item
  A more detailed \textbf{regression model} will be used to control for
  multiple factors affecting home prices, such as \textbf{income levels,
  employment rates, and local economic indicators}.
\end{itemize}

\paragraph{2. Panel Data Estimation}\label{panel-data-estimation}

\begin{itemize}
\tightlist
\item
  This method will allow for \textbf{cross-sectional and time-series
  analysis}, helping isolate the impact of different regional factors on
  housing prices.
\end{itemize}

\begin{center}\rule{0.5\linewidth}{0.5pt}\end{center}

\subsection{References}\label{references}

\begin{itemize}
\tightlist
\item
  \textbf{TIGER/Line Shapefiles:}
  \href{https://www2.census.gov/geo/tiger/TIGER2024/}{U.S. Census
  Bureau}
\item
  \textbf{Choropleth Mapping in R:}
  \href{https://r-graph-gallery.com/327-chloropleth-map-from-geojson-with-ggplot2.html}{R
  Graph Gallery}
\item
  \textbf{Customizing Legends in ggplot2:}
  \href{https://www.geeksforgeeks.org/control-size-of-ggplot2-legend-items-in-r/}{GeeksforGeeks}
\item
  \textbf{National Association of Realtors (2024)}
\item
  \textbf{Federal Reserve Bank of St.~Louis (FRED) (2024)}
\item
  \textbf{U.S. Census Bureau Population Estimates (2024)}
\item
  \textbf{Zillow Research Data (2024)}
\end{itemize}

\end{document}
